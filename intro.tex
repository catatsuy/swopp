プロセッサなどのハードウェア設計は,アーキテクチャ設計・論理設計・回路設計・物理設計といったフローで行われる.
アーキテクチャ設計と論理設計においては,RTL (Register Transfer Level) のシミュレーションが不可欠である.
このために Verilog HDL などのハードウェア記述言語が用いられることが一般的である.

そこで我々は C++ 言語上で RTL モデリングを行う新しい言語である ArchHDL を提案している \cite{satos:archhdl}.
それは Verilog HDL に近い記述でハードウェアの論理検証を行うことができる.

利点を次に挙げる.

\begin{itemize}
\itemsep1pt\parskip0pt\parsep0pt
\item
  ハードウェアモジュール間の接続の記述が容易
\item
  論理シミュレーションが高速
\end{itemize}

ArchHDL はオープンソースの Verilog シミュレータである Icarus Verilog \cite{iverilog}と比較して高速である.
しかし一部のハードウェアシミュレーションにおいて有償の Verilog シミュレータである NC-Verilog~\cite{ncverilog} より高速であったが,同じく有償の VCS~\cite{vcs} より高速ではないことがあった.

そこで分岐の削減,メモリ配置の工夫,OpenMP による並列化といった高速化手法を ArchHDL に適用できることを突き止めた.

本論文では Verilog シミュレータの Icarus Verilog/NC-Verilog/VCS と比較して,ArchHDL と今回の高速化手法がどの程度高速なのか考察する.

本論文の構成は以下の通りである. \ref{s:summary} 章で,ArchHDL の概要を述べる.
\ref{s:method} 章で, ArchHDL のプロファイリングとその結果から高速化手法を提案する.
\ref{s:evaluation} 章で,様々な Verilog シミュレーションツールと比較して ArchHDL と今回の高速化手法の評価を行う.
\ref{s:conclusion} 章でまとめる.