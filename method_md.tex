## 最適化の方針

最適化の方針として \ref{ss:profiling} 章逐次プログラミングにおける最適化と並列化における最適化の両方を考える.


## 逐次プログラミングにおける高速化手法

\ref{ss:profiling} 節で述べたように,ArchHDL::Module クラスの Always() メソッドと
reg::Update() メソッドに時間がかかっている.このメソッドを高速化出来ないのか考える.


### set_ 変数を無くす \label{s:no_set}

\figref{src:reg} に記すように,ver1.0 の ArchHDL の実装では reg インスタンス生成時に set_ を false
にしておき,ノンブロッキング代入が行われたときだけ true にすることで
reg::Update() 内の next_ の値を curr_
に代入する処理が next_ の値に変更があった時のみ行われるようにしている.

この set_ 変数を無くすと分岐と変数が 1 つ減る.よってノンブロッキング代入が頻繁に行われる回路では速度が向上すると考えられる.この実装を行ったものを ver1.1 とした.



### ダブルバッファリング

\ref{ss:implementation} 章で述べたように
reg インスタンスの next_ の値が次サイクルになる前に reg インスタンスの curr_ に代入される.
そこで偶数回目の実行と奇数回目の実行で next_ と curr_
を入れ替えれば(ダブルバッファリング)代入が減ることが期待できる.

\begin{figure}[t]
 \begin{center}
  \setlength{\unitlength}{1truemm} %picture環境の単位が1mmになる
\begin{picture}(60,24)(0,-15)
 \qbezier(2.5,-5)(0,0)(2.5,5)
 \put(2.5,5){\vector(1,2){1}}
 \put(2.5,-5){\vector(1,-2){1}}
 % cycle1
 \put(5,6){\texttt{cycle1}\hspace{15pt}\texttt{vala\_ (curr\_)}\hspace{5pt} $\rightarrow$ \hspace{5pt} \texttt{read}}
 \put(5,1){\phantom{\texttt{cycle1}}\hspace{15pt}\texttt{valb\_ (next\_)}\hspace{5pt} $\leftarrow$ \hspace{5pt} \texttt{write}}
 
 % cycle2
 \put(5,-8){\texttt{cycle2}\hspace{15pt}\texttt{vala\_ (next\_)}\hspace{5pt} $\leftarrow$ \hspace{5pt} \texttt{write}}
 \put(5,-13){\phantom{\texttt{cycle2}}\hspace{15pt}\texttt{valb\_ (curr\_)}\hspace{5pt} $\rightarrow$ \hspace{5pt} \texttt{read}}
\end{picture}
 \end{center}
 \caption{ダブルバッファリングのイメージ}
 \label{fig:double_buffer}
\end{figure}

\figref{fig:double_buffer} はダブルバッファリングのイメージである.つまり cycle1 では vala_ を curr_ として使用する.valb_ は next_ として使用する.サイクルが終了したら cycle2 に切り替える.これにより cycle1 の時点で next_ であった valb_ には値が書き込まれているので curr_ として使用出来る.これにより毎サイクル行われる代入処理を減らせる.

しかしこのダブルバッファリングでは reg への書き込みが行われなかった場合に
reg のその時の next_ の値に更新が入らない.次のサイクルではその next_ がそのまま
curr_ として使用されるので古い値が使われてしまう.そのため単純な実装では Verilog HDL と同じ挙動にならない.

しかし常に reg への書き込みが行われるカウンター回路で簡易的な実装で試したところ効果があった.
なので並列化しない場合は高速化の手法として使える可能性がある.

しかし OpenMP で Update() メソッドの処理を並列化できないので得策ではない可能性が高い.

そのためダブルバッファリングによる評価は今回行わなかった.


### next_ から curr_ への代入をメモリーコピーにする

\ref{s:no_set} 節で述べた実装を行うと
Update() メソッド内で行なっているすべての reg の next_
の値を curr_ への代入は毎サイクル実行されることになる.

この代入と Update() メソッド自体の関数呼び出しの
2 つオーバーヘッドが ArchHDL の高速化を妨げている要因として考えられる.

そこで全 reg インスタンスの curr_, next_
の値全てを配列として持つようにする.こうすることにより毎サイクル行われる next_ から curr_
への代入がメモリをコピーするだけで済むような最適化がコンパイラによって行われる.
そのため代入する処理が高速になることが期待される.
また関数呼び出しもなくなるので関数呼び出しのオーバーヘッドもなくなることが期待できる.

そこで十分大きな配列を 2 つ用意してそれぞれに curr_, next_ の値を格納するようにする.
reg インスタンスは curr_, next_ の実体は持たず,ポインタとして持つ.

この実装を行ったものを ver2.0 とした.




## 並列化による高速化 \label{ss:parallel}

\figref{src:class_singleton} に示すように
ArchHDL の Singleton クラスの Exec() メソッド内の
for 文内で Module::Always() メソッドと
reg::Update() メソッドは毎サイクル,各クラスの全インスタンスで呼び出されている.
この for 文を OpenMP により並列化できることが期待できる.

\begin{figure}[t]
 \lstinputlisting[language=c++]{src/exec_openmp.cc}
 \caption{Exec メソッド内の for 文を OpenMP で並列化したプログラム}
 \label{src:exec_openmp}
\end{figure}

並列化されるように OpenMP の指示句を与えたものが \figref{src:exec_openmp} である.

OpenMP にはオプションとしてチャンクサイズなどを指定できる.デフォルトはスケジュールタイプが
static でチャンクはなしである.今回の ArchHDL の場合はスケジュールタイプを dynamic
にしたり,チャンクサイズを指定するとデフォルトよりも遅くなることが確認されている.

よって ArchHDL の場合はスケジュールタイプが static でチャンクはなしというデフォルトの挙動が効率が良いと考えられる.

