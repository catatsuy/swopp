本章では Icarus Verilog/NC-Verilog~\cite{ncverilog}/VCS~\cite{vcs}
での論理シュミレーションの実行時間と ArchHDL
での実行時間を比較し,評価する.

\begin{table}[t]
 \caption{実行環境一覧}
 \label{table:exec_env}
 \begin{center}
  \begin{tabular}{lcc} \toprule
         &  Icarus Verilog/ArchHDL  &  NC-Verilog/VCS   \\ \midrule
  OS     &  Ubuntu12.04             &  CentOS5.9        \\
  カーネル &  Linux 3.2.0-39-generic  &  Linux 2.6.18-348.4.1.el5   \\
  CPU    &  \multicolumn{2}{c}{Intel Core i7-3770K CPU @ 3.50GHz}   \\
  メモリ  &  \multicolumn{2}{c}{$16\,\mathrm{GB}$}  \\ \bottomrule
  \end{tabular}
 \end{center}
\end{table}

\tabref{table:exec_env} は Icarus Verilog/ArchHDL と NC-Verilog/VCS
の実行環境の一覧である.

両者の実行環境が異なる理由は NC-Verilog と VCS が RedHat 系しか実行環境が用意されていない.
また有償ツールであるためソースコードが公開されていない.そのためコンパイルすることもできない.
しかし CentOS5.9 は gcc のバージョンが 4.1.2 である.
\ref{ss:modeling}節で述べたように ArchHDL では C++11 のラムダ関数を用いて記述するため gcc のバージョンは 4.5 以上が必要である.
これらの事情によりスペックは同じだが異なる OS を使用して評価することにした.

今回の評価では現実的なハードウェアとしてステンシル計算回路を用いる.
またカウンター回路と XORSHIFT による乱数生成器をマイクロベンチマークとして用いる.

\subsection{ステンシル計算回路による評価}

\if0

\begin{table}[t]
 \caption{ステンシル計算回路でのプロファイリング結果 1.1}
 \label{table:stencil_prof1.1}
 \begin{center}
  % \setlength{\tabcolsep}{3pt}
  \begin{tabular}{lr} \toprule
  関数名 & 実行時間に占める割合 (\%) \\ \midrule
  reg::Update() (合計) & 16.57 \\
  ArchHDL::Step() & 12.47 \\
  brk & 15.05 \\ \bottomrule
  \end{tabular}
 \end{center}
\end{table}

\fi

\begin{figure}[t]
 \centering
 \includegraphics[clip,width=\linewidth]{stencil}
 \caption{ステンシル計算回路の Icarus Verilog と比較した実行時間の速度向上比}
 \label{fig:stencil}
\end{figure}

\figref{fig:stencil} はステンシル計算回路での実行結果である.

縦軸は Icarus Verilog と比較したそれぞれの速度向上比である.

OpenMP による並列化はスレッド数を 8 個にして計測している.

ステンシル計算回路の場合は Update() は 325,469,175
回呼ばれているのに対して, reg の値が更新されるのは 320,323,415
回であり, reg の値に更新がないのは 5,145,760
回である.つまり更新がないのは全体の約 $1.58\%$ 程度に過ぎない.そのため
\ref{sss:no_set}
節で述べたデータ変更の有無による条件分岐の除去を行った方が高速になる.

また Update() は 325,469,175
回呼ばれているのでこのメソッド呼び出しを減らし,かつ代入をメモリーコピーにする
\ref{sss:mem_copy}
節の逐次代入をメモリーコピーにする方が高速になっている.

また Module が 133 個,reg が 991
個存在する回路なので並列化の効果も大きい.

NC-Verilog は ArchHDL より高速でない.VCS は ArchHDL と set\_
変数なしの実装より高速であるが,メモリーコピーにする実装よりは高速でない.また並列化を行ったものより高速でない.

\subsection{マイクロベンチマークによる評価}

\begin{figure}[t]
 \centering
 \includegraphics[clip,width=\linewidth]{counter_con}
 \caption{カウンター回路の Icarus Verilog と比較した実行時間の速度向上比}
 \label{fig:counter_con}
\end{figure}

$n$ 個のカウンター回路を実行するもの.

(一定数超えると並列化の効果が出てくる)

\begin{figure}[t]
 \centering
 \includegraphics[clip,width=\linewidth]{xorshift}
 \caption{XORSHIFT による乱数生成器の Icarus Verilog と比較した実行時間の速度向上比}
 \label{fig:xorshift}
\end{figure}

\figref{fig:xorshift} は XORSHIFT による乱数生成器での実行結果である.試行回数は 524,288 回である.初期値の異なる乱数生成器を 512 個用意している.

縦軸は Icarus Verilog と比較したそれぞれの速度向上比である.




\subsection{高速化の解析}
