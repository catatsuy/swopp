\AtBeginDvi{\special{papersize=210truemm,297truemm}}
\usepackage[sc]{mathpazo}
\usepackage[scaled]{helvet}
\usepackage[scaled]{beramono}
\usepackage[bold]{otf}
\usepackage[dvipdfmx]{graphicx}
\usepackage[dvipdfmx,pstarrows]{curve2e}%{pict2e}
\usepackage{textcomp,mediabb,booktabs,ebezier}
\usepackage[final]{listings}
\lstset{                 %listingsの設定
numbers=left,            %行番号を左
numberstyle=\scriptsize, %
stepnumber=1,            %1行おきに行番号を
numbersep=1zw,           %ソースと行番号の間隔
lineskip=-0.5zw,         %行間隔 要調整
basicstyle=\ttfamily,     %ttfamily
xleftmargin=15pt,
}
\usepackage{caption,tabularx} 
\captionsetup[table]{skip=.5\baselineskip}
\usepackage{jlisting}
%\newcommand{\unit}[1]{\ifmmode\mathrm{\,[#1]}\else $\mathrm{[#1]}$\fi}%単位に使用 数式中では空白を本文中では空白なし
\graphicspath{{img/}}
\title{ArchHDLで記述したハードウェアの\\ 論理シュミレーションの高速化}

\affiliate{TOKYOTECH_B}{東京工業大学 工学部情報工学科\\
Department of Computer Science, Tokyo Institute of Technology
}

\affiliate{TOKYOTECH}{東京工業大学 大学院情報理工学研究科\\
Graduate School of Information Science and Engineering, Tokyo Institute of Technology
}

\author{金子 達哉}{Kaneko Tatsuya}{TOKYOTECH_B}[kaneko@arch.cs.titech.ac.jp]
\author{佐藤 真平}{Sato Shimpei}{TOKYOTECH}[satos@arch.cs.titech.ac.jp]
\author{吉瀬 謙二}{Kise Kenji}{TOKYOTECH}[kise@cs.titech.ac.jp]

\setcounter{巻数}{54}%vol53=2012
\setcounter{号数}{2}
\setcounter{page}{1}

% color
\definecolor{lightgray}{gray}{.8}