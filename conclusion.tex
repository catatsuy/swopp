ハードウェアの RTL モデリングのための新しい言語として提案している ArchHDL の高速化手法を提案し,実装し,評価した.
ArchHDL ではハードウェアのレジスタを変数,ワイヤを関数として扱うことで,C++ で RTL モデリングを実現する.

高速化手法として (1)データ変更の有無による条件分岐の除去,(2)値を配列として格納しメモリ配置を工夫,
(3)並列化手法を提案した.

提案手法を実装し,ArchHDL を Icarus Verilog と商用ツールである VCS, NC-Verilog の実行時間と比較した.
マイクロベンチマークである 4096 個のカウンタ回路を用いた評価では VCS より 56.7 倍高速であった.
現実的なハードウェアであるステンシル計算回路を用いた評価では VCS より 1.83 倍高速であった.
我々が知る限り最速な Verilog シミュレータである VCS よりも ArchHDL が高速にハードウェアシミュレーションが行えることを明らかにした.
