ハードウェアの RTL モデリングのための新しい言語として提案している ArchHDL を高速化する手法を述べた.
ArchHDL ではハードウェアのレジスタを変数,ワイヤを関数として扱うことで,C++ で RTL モデリングを実現する.

高速化手法としてポインタ参照を減らして逐次実行での実行時間を減少させた.
並列化が可能な部分があることも述べた.
ある程度規模の大きい回路で並列化とポインタ参照を減らす高速化を同時に適用することで最も高速にシュミレーションできることを述べた.

そして ArchHDL を Icarus Verilog と商用ツールである VCS, NC-Verilog の実行時間と比較した.
カウンター回路,乱数生成器とステンシル計算回路のシュミレーション速度が ArchHDL が最も高速であることを確認した.
