\documentclass[submit,techreq,noauthor,papersize,]{ipsj}
\usepackage[T1]{fontenc}
\usepackage[utf8]{inputenc}
\usepackage{lmodern}
\usepackage{amssymb,amsmath}
\usepackage{calc}
\AtBeginDvi{\special{papersize=210truemm,297truemm}}
\usepackage[sc]{mathpazo}
\usepackage[scaled]{helvet}
\usepackage[scaled]{beramono}
\usepackage[bold]{otf}
\usepackage[dvipdfmx]{graphicx}
\usepackage[dvipdfmx,pstarrows]{curve2e}%{pict2e}
\usepackage{textcomp,mediabb,booktabs,ebezier}
\usepackage[final]{listings}
\lstset{                 %listingsの設定
numbers=left,            %行番号を左
numberstyle=\scriptsize, %
stepnumber=1,            %1行おきに行番号を
numbersep=1zw,           %ソースと行番号の間隔
lineskip=-0.5zw,         %行間隔 要調整
basicstyle=\ttfamily,     %ttfamily
xleftmargin=15pt,
}
\usepackage{caption,tabularx} 
\captionsetup[table]{skip=.5\baselineskip}
\usepackage{jlisting}
%\newcommand{\unit}[1]{\ifmmode\mathrm{\,[#1]}\else $\mathrm{[#1]}$\fi}%単位に使用 数式中では空白を本文中では空白なし
\graphicspath{{img/}}
\title{ArchHDLで記述したハードウェアの\\ 論理シミュレーションの高速化}

\affiliate{TOKYOTECH_B}{東京工業大学 工学部情報工学科\\
Department of Computer Science, Tokyo Institute of Technology
}

\affiliate{TOKYOTECH}{東京工業大学 大学院情報理工学研究科\\
Graduate School of Information Science and Engineering, Tokyo Institute of Technology
}

\author{金子 達哉}{Kaneko Tatsuya}{TOKYOTECH_B}% [kaneko@arch.cs.titech.ac.jp]
\author{佐藤 真平}{Sato Shimpei}{TOKYOTECH}% [satos@arch.cs.titech.ac.jp]
\author{吉瀬 謙二}{Kise Kenji}{TOKYOTECH}% [kise@cs.titech.ac.jp]

\setcounter{巻数}{54}%vol53=2012
\setcounter{号数}{2}
\setcounter{page}{1}

% color
\definecolor{lightgray}{gray}{.8}

\author{}
\date{}

\begin{document}

\begin{abstract}
 本稿では, RTL モデリングのための新しい言語として提案された ArchHDL の高速化に取り組む.
ArchHDL はハードウェアのレジスタを変数,ワイヤを関数として扱うことで, C++ で Verilog HDL に近い
RTL モデリングを実現している.
従来から論理シミュレーションを高速に行うことが出来た言語であるが,今回はそれを更に高速化し,他のシミュレーションツールと比較してどの程度高速なのか述べる.
\end{abstract}

\maketitle

\section{はじめに}

プロセッサなどのハードウェア設計は,アーキテクチャ設計・論理設計・回路設計・物理設計といったフローで行われる.
アーキテクチャ設計と論理設計においては,RTL (Register Transfer Level) のシミュレーションが不可欠である.
このために Verilog HDL などのハードウェア記述言語が用いられることが一般的である.

そこで我々は C++ 言語上で RTL モデリングを行う新しい言語である ArchHDL を提案している \cite{satos:archhdl}.
それは Verilog HDL に近い記述でハードウェアの論理検証を行うことができる.

利点を次に挙げる.

\begin{itemize}
\itemsep1pt\parskip0pt\parsep0pt
\item
  ハードウェアモジュール間の接続の記述が容易
\item
  論理シミュレーションが高速
\end{itemize}

ArchHDL はオープンソースの Verilog シミュレータである Icarus Verilog \cite{iverilog}と比較して高速である.
しかし一部のハードウェアシミュレーションにおいて有償の Verilog シミュレータである NC-Verilog~\cite{ncverilog} より高速であったが,同じく有償の VCS~\cite{vcs} より高速ではないことがあった.

そこで分岐の削減,メモリ配置の工夫,OpenMP による並列化といった高速化手法を ArchHDL に適用できることを突き止めた.

本論文では Verilog シミュレータの Icarus Verilog/NC-Verilog/VCS と比較して,ArchHDL と今回の高速化手法がどの程度高速なのか考察する.

本論文の構成は以下の通りである. \ref{s:summary} 章で,ArchHDL の概要を述べる.
\ref{s:method} 章で, ArchHDL のプロファイリングとその結果から高速化手法を提案する.
\ref{s:evaluation} 章で,様々な Verilog シミュレーションツールと比較して ArchHDL と今回の高速化手法の評価を行う.


\section{ArchHDL の概要}

\label{s:summary}

\input{summary}

\section{ArchHDL の高速化手法の提案と実装}

\label{s:method}

\subsection{最適化の方針}

ここでは \ref{ss:profiling} 章で述べたように,ArchHDL::Step() と
reg::Update() の処理を高速化することを考える.

最適化の方針として逐次プログラミングにおける最適化と並列化における最適化の両方を考える.

\subsection{逐次プログラムにおける高速化手法}

\subsubsection{データ変更の有無による条件分岐の除去 \label{sss:no_set}}

\figref{src:reg} に示した実装では,reg
クラスのインスタンスの値を更新する方法としてブロッキング代入とノンブロッキング代入の
2 つが存在する.

まずノンブロッキング代入について考える.reg
クラスのインスタンスにノンブロッキング代入が行われた時にメンバ変数 set\_
を true にし,メンバ変数 next\_ に値を代入する.そして reg::Update
内では set\_ が true の時だけ next\_ をメンバ変数 curr\_
に代入する.これは reg
クラスのインスタンスの値を変更したサイクルのみで,その reg
クラスのインスタンスの値を次サイクルに移る前に新しい値に更新することを意味する.

\figref{src:reg} に示した実装では,更新されない reg
クラスのインスタンスの curr\_ と next\_
の値が同じであるため,代入する処理を行う必要はない.よって set\_
変数を用いて不要な代入を避けている.reg
クラスのインスタンスの更新頻度が低い回路であればこの実装が効率的である.

次にブロッキング代入について考える.reg
クラスのインスタンスにブロッキング代入が行われた時に curr\_
の値を書き換える.

提案手法について述べる.この set\_ 変数が true
の時のみ代入するのではなく,次サイクルに移る前に next\_ の値を curr\_
に常に代入するようにする.こうすることによって分岐のオーバーヘッドが無くなるため,ノンブロッキング代入が頻繁に行われる回路で速度向上が期待できる.

\subsubsection{値を配列として格納しポインタ参照を削減} \label{sss:mem_copy}

\begin{figure}[t]
 \centering
 \includegraphics[clip,width=\linewidth-30pt]{simple_reg}
 \caption{シンプルな実装による reg クラスのインスタンスの処理の様子}
 \label{fig:regs}
\end{figure}

\figref{fig:regs} はシンプルな実装による reg クラスのインスタンスの処理の様子である.\figref{src:class_singleton}
の 44 行から 46 行の処理を表している.reg クラスのインスタンスが灰色に塗られており,左からクラスのメタデータ,next\_,curr\_
を表している.左側の大きな枠が示した実装の \verb`std::vector` 型の registers\_ である.
実線矢印は代入を表し.点線矢印はポインタ参照を表す.

registers\_ の値を上から順に辿る.最初の reg クラスのインスタンスの reg::Update() メソッドを呼び,next\_ の値が curr\_ に代入される.これを全 reg クラスのインスタンスにおいて行う.

\ref{sss:no_set} 節で述べたデータ変更の有無による条件分岐の除去を行うと
reg::Update() メソッド内で行なっている reg
クラスのインスタンスの curr\_ に next\_ の値を代入する処理は毎サイクル全
reg クラスのインスタンスで実行されることになる.

この代入する処理と reg::Update() メソッド自体の関数呼び出しの 2
つのオーバーヘッドが\tabref{table:stencil_prof} で示すようにArchHDL
の高速化を妨げている.

\begin{figure}[t]
 \centering
 \includegraphics[clip,width=\linewidth-30pt]{mem_copy}
 \caption{値を配列として格納しポインタ参照を削減するようにした reg クラスのインスタンスの処理の様子}
 \label{fig:mem_copy}
\end{figure}

\figref{fig:mem_copy} に値を配列として格納しポインタ参照を削減するようにした reg クラスのインスタンスの処理の様子を示す.reg クラスのインスタンスが灰色に塗られており,左からクラスのメタデータ,next\_,curr\_ を表している.
下の枠が next\_ と curr\_ の値をまとめた配列であり,ここでは next collections, curr collections と呼ぶ.
実線矢印は代入を表し.点線矢印はポインタ参照を表す.

全 reg クラスのインスタンスは現在の値と次サイクルの値の実体は持たず,ポインタを保持するようにする.実体はそれぞれ配列として持つ.

次サイクルに移る前に行われる curr\_ に next\_ の値を代入する処理は \figref{fig:regs} で示すようにシンプルな実装では registers\_ から reg クラスのインスタンスが存在するアドレスを調べる必要がある.
しかし値を配列として格納しポインタ参照を削減するようにすると \figref{fig:mem_copy} に示すように単純なメモリの並びを代入するだけになる.これにより代入する処理が高速になることが期待される.

またこの手法によって reg::Update() 自体の関数呼び出しがなくなり,関数呼び出しのオーバーヘッドもなくなる.

実装は 2 つの大きな配列を用意する.その配列内に reg クラスのインスタンス宣言時に記述した型に応じた領域を確保する.
reg クラスのコンストラクタで next\_ と curr\_ の値が存在するアドレスを取得し,インスタンスでそれを保持する.
これまで reg クラスの全インスタンスの reg::Update() メソッドを呼び出していたところを next\_ collections から curr\_ collections の値コピーに変更する.


\subsubsection{ダブルバッファリング}

これまでの実装では \ref{ss:implementation} 章で述べたように reg
クラスのインスタンスの次サイクルの値が次サイクルに移る前に reg
クラスのインスタンスの現在の値に代入される.
そこで偶数回目の実行と奇数回目の実行で次サイクルの値と現在の値を格納している変数を
を入れ替えれば(ダブルバッファリング)代入が減ることが期待できる.

\begin{figure}[t]
 \begin{center}
  \setlength{\unitlength}{1truemm} %picture環境の単位が1mmになる
\begin{picture}(60,13)(0,-3)
 \put(10,0){\framebox(10,10){\texttt{next\_}}}
 \put(40,0){\framebox(10,10){\texttt{curr\_}}}
 \put(0,5){\vector(1,0){8}}
 \put(0,7){\texttt{write}}
 \put(22,7){\texttt{every cycle}}
 \put(22,5){\vector(1,0){16}}
 \put(22,2){\texttt{\phantom{sss}write}}
 \put(52,5){\vector(1,0){8}}
 \put(52.5,7){\texttt{read}}
\end{picture}
 \end{center}
 \caption{reg クラスのインスタンスの変数保持の処理の様子}
 \label{fig:reg_curr_next}
\end{figure}

\begin{figure}[t]
 \begin{center}
  \setlength{\unitlength}{1truemm} %picture環境の単位が1mmになる
\begin{picture}(85,33)(0,-3)
 \put(35,20){\framebox(10,10){}}
 \put(55,20){\framebox(10,10){}}
 \put(35,0){\framebox(10,10){}}
 \put(55,0){\framebox(10,10){}}
 \put(25,25){\vector(1,0){8}}
 \put(25,27){\texttt{write}}
 \put(67,25){\vector(1,0){8}}
 \put(67.5,27){\texttt{read}}
 % bottom
 \put(33,5){\vector(-1,0){8}}
 \put(25.5,7){\texttt{read}}
 \put(75,5){\vector(-1,0){8}}
 \put(67.5,7){\texttt{write}}
 % round arrow
 \qbezier(2.5,8)(0,15)(2.5,22)
 \put(2.5,22){\vector(1,2){1}}
 \put(2.5,8){\vector(1,-2){1}}
 % cycle
 \put(5,23){\texttt{odd cycle}}
 \put(5,6){\texttt{even cycle}}
\end{picture}
 \end{center}
 \caption{ダブルバッファリングの処理の様子}
 \label{fig:double_buffer}
\end{figure}

\figref{fig:reg_curr_next} はこれまでの ArchHDL の reg
クラスのインスタンスの値の保持のイメージである.読み込み用と書き込み用の変数をそれぞれ保持している.読み込み用が現在の値であり,書き込み用が次サイクルの値である,次サイクルに移る前に書き込み用の値が読み込み用の変数に書き込まれる.

\figref{fig:double_buffer}
はダブルバッファリングのイメージである.奇数回目のサイクルと偶数回目のサイクルで読み込み用と書き込み用の変数を入れ替える.これにより奇数回目のサイクルで書き込み用であった変数には値が書き込まれているので次サイクルの偶数回目のサイクルで読み込み用として使用出来る.これを繰り返すことで,次サイクルに移る前に行われる代入処理を無くせる.

しかし今回の手法では reg
クラスのインスタンスへ値の書き込みが行われなかった場合に reg
クラスのインスタンスのその時の書き込み用の値に更新が入らない.次サイクルではその書き込み用の値がそのまま現在の値として使用されるので古い値が使われてしまう.そのため単純に入れ替えるだけの実装では誤ったシュミレーションを行なってしまう.

また今回の手法はサイクルの回数で依存関係が発生するので \ref{ss:parallel}
節で述べる並列化ができない.

よってライブラリの実装として導入するのは困難であるが,reg
クラスのインスタンスへ常に書き込みが行われるカウンター回路で試したところ効果があった(具体的な数字).常に
reg
クラスのインスタンスに書き込みが行われるハードウェアシュミレーションを逐次処理で行いたい場合には高い効果が期待できる.

以上の理由から本論文ではダブルバッファリングによる評価は行わない.

\subsection{並列化による高速化 \label{ss:parallel}}

一方で並列化による高速化では \figref{src:class_singleton} に示すように
ArchHDL の Singleton クラスの Exec() メソッド内の for 文内で
Module::Always() メソッドと reg::Update()
メソッドは毎サイクル,各クラスの全インスタンスで呼び出されている.
モジュールの実行とレジスタの更新はそれぞれ独立に行えるので容易に並列化が可能である.

今回は for 文などの並列化を行いたい部分の前に \verb/#pragma omp/
から始まる OpenMP
指示文を与えるだけで並列化プログラミングを行える便利なライブラリである
OpenMP \cite{openmp}を使用する.

\begin{figure}[t]
 \lstinputlisting[language=c++]{src/exec_openmp.cc}
 \caption{Exec メソッド内の for 文を OpenMP で並列化したプログラム}
 \label{src:exec_openmp}
\end{figure}

8 スレッドで並列化されるように OpenMP 指示文を与えたものが
\figref{src:exec_openmp} である.

一般に並列化を行う場合は各スレッド対して均等に負荷を与えることが重要である.for
文の負荷を分散するスケジュール方法として静的に決定する static
と,動的に決定する dynamic など複数存在する.今回の ArchHDL
の場合,各モジュールと各レジスタで実行時間が大幅に変わるというのは考えにくい.よって動的に決定するオーバーヘッドを考えると
static を指定した方が効率が良いと考えられる.実際に dynamic
を指定すると遅くなることが確認されている.

他にも OpenMP
にはオプションとしてチャンクサイズ(割り当てサイズ)を指定できる.チャンクサイズを指定するとデフォルトよりも遅くなることが確認されている(具体的な数字を載せるべき?).

よって今回の評価はスケジュール方法が static
でチャンクサイズは指定しないというデフォルトの設定で行う.


\section{評価}

\label{s:evaluation}

本章では ArchHDL での論理シミュレーションの実行時間を評価し,Icarus Verilog, NC-Verilog, VCS での論理シミュレーションの実行時間と比較する.

\begin{table}[t]
 \caption{実行環境}
 \label{table:exec_env}
 \begin{center}
  \begin{tabular}{l|c|c} \hline
         &  Icarus Verilog, ArchHDL  &  NC-Verilog, VCS   \\ \hline
  OS     &  Ubuntu12.04             &  CentOS5.9        \\
  CPU    &  Core i7-3770K 3.50GHz   &  Core i7-3770K 3.50GHz  \\
  メモリ  &  $16\,\mathrm{GB}$       &  $16\,\mathrm{GB}$  \\ \hline
  \end{tabular}
 \end{center}
\end{table}

\tabref{table:exec_env} に実行環境をまとめる.
評価には同じ仕様の 2 台の計算機を用いる.
一台は Icarus Verilog, ArchHDL の評価に用いる.もう一台は NC-Verilog, VCS の評価に用いる.
CPU,メモリなどのハードウェアの仕様は同一であるがソフトウェアの制約により異なる OS を利用する.

異なる OS を用いる理由を述べる.
NC-Verilog と VCS は RedHat 系のディストリビューションのみをサポートしている.
今回は RedHat 系のディストリビューションである CentOS5.9 を用いる.
しかし CentOS5.9 に含まれる gcc のバージョンは 4.1.2 である.
\ref{ss:modeling}節で述べたように,ArchHDL では C++11 のラムダ関数を用いて記述するため gcc のバージョンは 4.5 以上が必要である.
その条件を満たす Ubuntu12.04 を評価に用いる.
Ubuntu12.04 に含まれる gcc のバージョンは 4.6.3 である.
gcc の最適化オプションとして \verb/-O2/ を用いる.
Icarus Verilog はどちらのディストリビューションでも動作するが,今回は Ubuntu12.04 を用いる.
Ubuntu12.04 に含まれる Icarus Verilog のバージョンは 0.9.5 である.

今回用いる計算機の CPU の物理コアは 4 コアであるため,OpenMP による並列化はスレッド数を 8 個にして評価する.

評価では,2 つのマイクロベンチマークと,現実的なハードウェアのベンチマークとしてステンシル計算回路\cite{koba:stencil}を用いる.
Verilog HDL と ArchHDL のためのハードウェアシミュレーションは手作業により作成した.
両ハードウェアシミュレーションの出力は同様になるように作成した.

評価結果に用いているラベルの名前について述べる.
オリジナルの ArchHDL は \textbf{ArchHDL} と表す.
\ref{sss:no_set} 節で述べた条件分岐の除去を適用したものを \textbf{NO SET} と表す.
\ref{sss:mem_copy} 節で述べたメモリ配置の工夫を適用したものを \textbf{MEM MAP} と表す.
\ref{ss:parallel} 節で述べた並列化を行ったものを \textbf{PARA} と表す.
メモリ配置の工夫と並列を同時に適用したものを \textbf{MEM MAP + PARA} と表す.


\subsection{マイクロベンチマークによる評価}

マイクロベンチマークとしてカウンタ回路と XORSHIFT による乱数生成回路を用いる.

\begin{figure}[t]
 \lstinputlisting[language=c++]{src/xorshift_alg.cc}
 \caption{XORSHIFT 法に基づく乱数生成のアルゴリズム}
 \label{src:xorshift_alg}
\end{figure}

カウンタ回路とは \figref{src:counter} に示した 1 サイクルごとに 1 を足す回路である.
ハードウェアの規模を増やすためにカウンタの数を指定できるようにした.
XORSHIFT による乱数生成回路とはシフトと XOR 演算のみで構成できる XORSHIFT 法に基づく乱数生成器をハードウェア記述によって実装した回路である.
\figref{src:xorshift_alg} に XORSHIFT 法に基づく乱数生成のアルゴリズムを C 言語によって実装したものを示す.

\begin{figure}[t]
 \centering
 \includegraphics[clip,width=\linewidth]{counter_4096}
 \caption{4096 個のカウンタ回路の実行時間を Icarus Verilog と比較した速度向上比}
 \label{fig:counter4096}
\end{figure}

\figref{fig:counter4096} に 4096 個のカウンタ回路の実行時間を Icarus Verilog と比較した速度向上比を示す.
縦軸は Icarus Verilog での実行時間を 1 とした速度向上比を示している.

ArchHDL は商用の NC-Verilog, VCS と比較してもかなり高速である.
\textbf{MEM MAP + PARA} の論理シミュレーション実行時間は NC-Verilog の 58.8 倍,VCS の 56.7 倍高速である.

また今回提案している高速化手法はオリジナルの ArchHDL に比べていずれも効果が出ている.
\textbf{MEM MAP + PARA} の論理シミュレーション実行時間はオリジナルの ArchHDL の 5.23 倍高速である.



\begin{figure}[t]
 \centering
 \includegraphics[clip,width=\linewidth]{counter_con}
 \caption{高速化手法を適用した ArchHDL と OpenMP を適用したカウンタ回路の実行時間を Icarus Verilog と比較した速度向上比}
 \label{fig:counter_con}
\end{figure}

\figref{fig:counter_con} に高速化手法を適用した ArchHDL と OpenMP を適用したカウンタ回路の実行時間を Icarus Verilog と比較した速度向上比を示す.
縦軸は Icarus Verilog での実行時間を 1 とした速度向上比を示している.
横軸はカウンタの個数である.

\textbf{MEM MAP} は逐次に実行されているので Icarus Verilog と比較した速度向上比はカウンタの個数を変えてもほとんど変わらない.
並列化を行った \textbf{PARA} と \textbf{MEM MAP + PARA} はカウンタの個数が 1024 個以上で \textbf{MEM MAP} よりも高速になる.
\textbf{PARA} より \textbf{MEM MAP + PARA} の方が常に高速であるので今回提案している逐次処理での高速化手法は並列化を行った場合でも効果が出ている.
カウンタの個数はハードウェアの規模とみなせるため,並列化が有効なのはある程度規模の大きい回路であると言える.


\begin{figure}[t]
 \centering
 \includegraphics[clip,width=\linewidth]{xorshift}
 \caption{512 個の XORSHIFT による乱数生成器の実行時間を Icarus Verilog と比較した速度向上比}
 \label{fig:xorshift}
\end{figure}

\figref{fig:xorshift} は XORSHIFT による乱数生成器での実行時間を Icarus Verilog と比較した速度向上比である.
試行回数は 524,288 回である.初期値の異なる乱数生成器を 512 個用意している.

ArchHDL は商用の NC-Verilog, VCS と比較してもかなり高速である.
\textbf{MEM MAP + PARA} の論理シミュレーション実行時間は NC-Verilog の 32.2 倍,VCS の 11.3 倍高速である.

また今回提案している高速化手法はオリジナルの ArchHDL に比べていずれも効果が出ている.
\textbf{MEM MAP + PARA} の論理シミュレーション実行時間はオリジナルの ArchHDL の 2.78 倍高速である.


\subsection{ステンシル計算回路による評価}

\if0

\begin{table}[t]
 \caption{ステンシル計算回路でのプロファイリング結果 1.1}
 \label{table:stencil_prof1.1}
 \begin{center}
  % \setlength{\tabcolsep}{3pt}
  \begin{tabular}{lr} \toprule
  関数名 & 実行時間に占める割合 (\%) \\ \midrule
  reg::Update() (合計) & 16.57 \\
  ArchHDL::Step() & 12.47 \\
  brk & 15.05 \\ \bottomrule
  \end{tabular}
 \end{center}
\end{table}

\fi

\begin{figure}[t]
 \centering
 \includegraphics[clip,width=\linewidth]{stencil}
 \caption{ステンシル計算回路の Icarus Verilog と比較した実行時間の速度向上比}
 \label{fig:stencil}
\end{figure}

\figref{fig:stencil} はステンシル計算回路での実行結果である.
縦軸は Icarus Verilog と比較したそれぞれの速度向上比である.

オリジナルの ArchHDL は商用の NC-Verilog より高速であったが,同じく商用の VCS はオリジナルの ArchHDL と \textbf{NO SET} より高速である.
しかし逐次実行での高速化手法と並列化を共に適用した \textbf{MEM MAP + PARA} の論理シミュレーション実行時間は VCS の 1.83 倍高速である.

ステンシル計算回路の場合は Update() は 325,469,175 回呼ばれているのに対して,
reg の値に更新がないのは 5,145,760 回である.
つまり更新がないのは Update() メソッド呼び出し全体の $1.58\%$ 程度に過ぎない.
それにより条件分岐を無くす \textbf{NO SET} の論理シミュレーションはオリジナルの ArchHDL より高速である.
また Update() のメソッド呼び出しを減らし,かつメモリ配置を工夫している \textbf{MEM MAP} の論理シミュレーション実行時間はオリジナルの ArchHDL の 1.31 倍高速である.

また Module が 133 個,reg が 991 個存在する回路なので並列化の効果も大きい.
逐次実行での高速化手法と並列化を共に適用した \textbf{MEM MAP + PARA} の論理シミュレーション実行時間はオリジナルの ArchHDL の 1.95 倍高速である.


% \subsection{高速化の解析}


\section{まとめ}

\label{s:conclusion}

ハードウェアの RTL モデリングのための新しい言語として提案している ArchHDL の高速化手法を提案し,実装し,評価した.
ArchHDL ではハードウェアのレジスタを変数,ワイヤを関数として扱うことで,C++ で RTL モデリングを実現する.

高速化手法として (1)データ変更の有無による条件分岐の除去,(2)値を配列として格納しポインタ参照を削減,
(3)並列化手法を提案した.

提案手法を実装し,ArchHDL を Icarus Verilog と商用ツールである VCS, NC-Verilog の実行時間と比較した.
マイクロベンチマークである 4096 個のカウンタ回路を用いた評価では VCS より 56.7 倍高速であった.
現実的なハードウェアであるステンシル計算回路を用いた評価では VCS より 1.83 倍高速であった.
我々が知る限り最速な Verilog シミュレータである VCS よりも ArchHDL が高速にハードウェアシミュレーションが行えることを明らかにした.


\section*{謝辞}

\label{s:acknowledgment}

研究を進めるにあたり.適切な指導をしていただいた指導教員の吉瀬謙二准教授に感謝します.吉瀬研究室の皆様にも数々の助言をいただき,大変お世話になりました.特に吉瀬研究室の博士課程の佐藤真平さんには多大な貢献をしていただきました.また同じく博士課程の高前田(山崎)伸也さんと笹河良介さんにも数多くの助言をいただきました.

ArchHDL の開発に多大な貢献をしていただいた佐野伸太郎さんに感謝致します.


\begin{thebibliography}{99}
 \bibitem{satos:archhdl}
   佐藤真平,吉瀬謙二:
   C++ をベースとする新しいハードウェア記述の検討,
   情報処理学会研究報告 (2013)
 \bibitem{iverilog}
   Icarus Verilog: http://iverilog.icarus.com/ .
 \bibitem{vcs}
   VCS http://www.synopsys.com/VCS
 \bibitem{ncverilog}
   NC-Verilog
 \bibitem{gprof}
   gprof http://www.gnu.org/software/binutils/
 \bibitem{openmp}
   OpenMP http://openmp.org/
\end{thebibliography}


\end{document}
