プロセッサなどのハードウェア設計は,アーキテクチャ設計・論理設計・回路設計・物理設計といったフローで行われる.
アーキテクチャ設計と論理設計においては,RTL (Register Transfer Level)
のシミュレーションが不可欠である.このために
Verilog HDL などのハードウェア記述言語が用いられることが一般的である.

そこで我々は C++ 言語上で RTL モデリングを行う新しい言語を提供する ArchHDL を提案している \cite{satos:archhdl}.
ArchHDL は Verilog HDL に近い記述でハードウェアの論理検証を行うことができる.

ArchHDL には次に挙げる利点がある.

* ハードウェアモジュール間の接続の記述が容易
* 論理シュミレーションが高速

ArchHDL は Icarus Verilog \cite{iverilog}と比較して高速である.

しかし一部のハードウェアシュミレーションにおいて有償の Verilog シュミレーションと比較して高速ではないこともあった.

そこで条件分岐を減らす,メモリ配置を工夫する,OpenMP によって一部を並列化する,と高速化ができることに着目し,従来の ArchHDL を更に高速化できることを突き止めた.

本論文では様々な Verilog シュミレーションツールと比較して, ArchHDL と今回の高速化手法がどの程度高速なのか考察する.


本論文の構成は以下の通りである. \ref{s:summary} 章で,ArchHDL の概要を述べる.
\ref{s:method} 章で, ArchHDL のプロファイリングとその結果から高速化手法を提案する.
\ref{s:evaluation} 章で,様々な Verilog シュミレーションツールと比較して ArchHDL と今回の高速化手法の評価を行う.


