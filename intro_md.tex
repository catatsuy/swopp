## 研究の背景と目的

プロセッサなどのハードウェア設計は,アーキテクチャ設計・論理設計・回路設計・物理設計といったフローで行われる.
アーキテクチャ設計と論理設計においては,RTL (Register Transfer Level)
のシミュレーションが不可欠である.このために
Verilog HDL などのハードウェア記述言語が用いられることが一般的である.

そこで C++ 言語上で RTL モデリングを行う新しい言語を提供する ArchHDL が提案されている \cite{satos:archhdl}.
ArchHDL は Verilog HDL に近い記述でハードウェアの論理検証を行うことができる.

ArchHDL では次に上げる利点がある.

* ハードウェアモジュール間の接続の記述が容易
* 論理シュミレーションが高速

ArchHDL は Icarus Verilog と比較して高速である.本論文では従来の
ArchHDL を改良して更に高速化し, Icarus Verilog と比較してどの程度高速なのか考察する.


## 本論文の構成

本論文の構成は以下の通りである. \ref{s:summary} 章で,今回高速化を行う ArchHDL の概要を述べる.
\ref{s:method} 章で, ArchHDL のプロファイリングとその結果から高速化手法を提案する.
\ref{s:evaluation} 章で, ArchHDL の評価を行う.


